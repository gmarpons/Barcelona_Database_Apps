\documentclass[]{article}
\usepackage[T1]{fontenc}
\usepackage{lmodern}
\usepackage{amssymb,amsmath}
\usepackage{ifxetex,ifluatex}
\usepackage{fixltx2e} % provides \textsubscript
% Set line spacing
% use upquote if available, for straight quotes in verbatim environments
\IfFileExists{upquote.sty}{\usepackage{upquote}}{}
\ifnum 0\ifxetex 1\fi\ifluatex 1\fi=0 % if pdftex
  \usepackage[utf8]{inputenc}
\else % if luatex or xelatex
  \ifxetex
    \usepackage{mathspec}
    \usepackage{xltxtra,xunicode}
  \else
    \usepackage{fontspec}
  \fi
  \defaultfontfeatures{Mapping=tex-text,Scale=MatchLowercase}
  \newcommand{\euro}{€}
\fi
% use microtype if available
\IfFileExists{microtype.sty}{\usepackage{microtype}}{}
\usepackage[margin=1in]{geometry}
\usepackage{longtable,booktabs}
\usepackage{graphicx}
% Redefine \includegraphics so that, unless explicit options are
% given, the image width will not exceed the width of the page.
% Images get their normal width if they fit onto the page, but
% are scaled down if they would overflow the margins.
\makeatletter
\def\ScaleIfNeeded{%
  \ifdim\Gin@nat@width>\linewidth
    \linewidth
  \else
    \Gin@nat@width
  \fi
}
\makeatother
\let\Oldincludegraphics\includegraphics
{%
 \catcode`\@=11\relax%
 \gdef\includegraphics{\@ifnextchar[{\Oldincludegraphics}{\Oldincludegraphics[width=\ScaleIfNeeded]}}%
}%
\ifxetex
  \usepackage[setpagesize=false, % page size defined by xetex
              unicode=false, % unicode breaks when used with xetex
              xetex]{hyperref}
\else
  \usepackage[unicode=true]{hyperref}
\fi
\hypersetup{breaklinks=true,
            bookmarks=true,
            pdfauthor={Grup d'Anàlisi de Dades},
            pdftitle={Perfil de barri o districte},
            colorlinks=true,
            citecolor=blue,
            urlcolor=blue,
            linkcolor=magenta,
            pdfborder={0 0 0}}
\urlstyle{same}  % don't use monospace font for urls
\setlength{\parindent}{0pt}
\setlength{\parskip}{6pt plus 2pt minus 1pt}
\setlength{\emergencystretch}{3em}  % prevent overfull lines
\setcounter{secnumdepth}{0}

%%% Change title format to be more compact
\usepackage{titling}
\setlength{\droptitle}{-2em}
  \title{Perfil de barri o districte}
  \pretitle{\vspace{\droptitle}\centering\huge}
  \posttitle{\par}
  \author{Grup d'Anàlisi de Dades}
  \preauthor{\centering\large\emph}
  \postauthor{\par}
  \predate{\centering\large\emph}
  \postdate{\par}
  \date{01/02/2015}




\begin{document}

\maketitle


\section{DISTRICTE de `SANT MARTÍ'}\label{districte-de-sant-marti}

\subsection{Dades polítiques}\label{dades-politiques}

\begin{longtable}[c]{@{}lll@{}}
\toprule\addlinespace
\textbf{\ldots{}} & \textbf{al districte} & \textbf{a la ciutat}
\\\addlinespace
\midrule\endhead
\\\addlinespace
Abstenció eleccions generals 2011 & 32.3 & 31.9
\\\addlinespace
\\\addlinespace
Abstenció eleccions locals 2011 & 48 & 47
\\\addlinespace
Vot a CIU eleccions locals 2011 & 21.7 & 28.2
\\\addlinespace
Vot a PSC eleccions locals 2011 & 25.7 & 21.8
\\\addlinespace
Vot a PP eleccions locals 2011 & 16.6 & 16.9
\\\addlinespace
Vot a ICV eleccions locals 2011 & 11.5 & 10.2
\\\addlinespace
Vot a ERC eleccions locals 2011 & 5.8 & 5.5
\\\addlinespace
\\\addlinespace
Vot a CIU eleccions europees 2014 & 14.9 & 20.7
\\\addlinespace
Vot a PSC eleccions europees 2014 & 14.4 & 12.1
\\\addlinespace
Vot a PP eleccions europees 2014 & 10.8 & 11.9
\\\addlinespace
Vot a ICV eleccions europees 2014 & 14.4 & 12.5
\\\addlinespace
Vot a ERC eleccions europees 2014 & 22.9 & 21.6
\\\addlinespace
Vot a Podemos eleccions europees 2014 & 5.6 & 4.7
\\\addlinespace
\\\addlinespace
\bottomrule
\end{longtable}

\subsection{Dades socio-econòmiques}\label{dades-socio-economiques}

\begin{longtable}[c]{@{}lll@{}}
\toprule\addlinespace
\textbf{\ldots{}} & \textbf{al districte} & \textbf{a la ciutat}
\\\addlinespace
\midrule\endhead
\\\addlinespace
Atur (estimació)\footnote{Disposem a nivell de barri de dades sobre la
  població i sobre el nombre d'aturats registrats. Per calcular la taxa
  d'atur cal saber la taxa d'activitat (no disponible per barris) i el
  nombre d'aturats real (més alt que els registrats). Per tant, cal fer
  una estimació aproximada. Usant com a referència dades de la EPA del
  4rt trimestre de 2013, hem multiplicat la població total per 0.51 per
  estimar la població activa i hem multiplicat l'atur registrat per 1.38
  per estimar la població aturada.} & 18.8 & 16.9
\\\addlinespace
\\\addlinespace
Renda familiar\footnote{Es tracta de la `renda familiar bruta
  disponible', un índex en base a 100 que calcula el servei
  d'estadística de l'Ajuntament.} 2014 & 81.2 & 100
\\\addlinespace
\\\addlinespace
\bottomrule
\end{longtable}

\subsection{Habitatge}\label{habitatge}

\begin{longtable}[c]{@{}lll@{}}
\toprule\addlinespace
\textbf{\ldots{}} & \textbf{al districte} & \textbf{a la ciutat}
\\\addlinespace
\midrule\endhead
\\\addlinespace
Preu metre quadrat compra 2013 & 2097 & 2454
\\\addlinespace
Preu metre quadrat lloguer 2011 & 11.7 & 11.9
\\\addlinespace
\\\addlinespace
Persones per llar & 2.5 & 2.4
\\\addlinespace
Densitat (habitants/hectàrea) & 221.3 & 157.8
\\\addlinespace
\\\addlinespace
Llars comprades i pagades (\%) & 41 & 37.6
\\\addlinespace
Llars de lloguer (\%) & 22 & 30.1
\\\addlinespace
Llars comprades hipotecades\footnote{El percentatge de llars amb
  diferent tipus de tinença no sumen 100 perquè hi ha altres tipus (ex.:
  `herència', `cedides', \ldots{}) no recollits en aquesta taula.} (\%)
& 28.5 & 22.5
\\\addlinespace
\\\addlinespace
Llars sense calefacció (\%) & 10.7 & 10.7
\\\addlinespace
Llars sense internet\footnote{Les dades sobre percentatges de llars amb
  manques (calefacció, internet) poden contenir errors perquè sovint hi
  ha sobre-estimació del nombre de llars (denominador), disminuint
  llavors el percentatge real.} (\%) & 34.8 & 34.6
\\\addlinespace
\\\addlinespace
\bottomrule
\end{longtable}

\subsection{Població}\label{poblacio}

\begin{longtable}[c]{@{}lll@{}}
\toprule\addlinespace
\textbf{\ldots{}} & \textbf{al districte} & \textbf{a la ciutat}
\\\addlinespace
\midrule\endhead
\\\addlinespace
Població a Barcelona el 2013 (en milers) & 233 & 1612
\\\addlinespace
Nascuts a Catalunya (\%) & 59.3 & 59.2
\\\addlinespace
Nascuts a Espanya fora de Catalunya (\%) & 24.7 & 23.3
\\\addlinespace
Nascuts a l'extranger\footnote{Els percentatges no sumen exactament 100
  per diferències entre les fonts. Són estimacions aproximades.} (\%) &
19.8 & 21.8
\\\addlinespace
\\\addlinespace
\bottomrule
\end{longtable}

\subsection{Dades socio-demogràfiques}\label{dades-socio-demografiques}

\begin{longtable}[c]{@{}lll@{}}
\toprule\addlinespace
\textbf{\ldots{}} & \textbf{al districte} & \textbf{a la ciutat}
\\\addlinespace
\midrule\endhead
\\\addlinespace
Taxa de natalitat 2014 & 9 & 8.2
\\\addlinespace
Taxa de mortalitat 2014 & 8.1 & 9.2
\\\addlinespace
Taxa bruta d'immigració 2014 & 42.9 & 47.1
\\\addlinespace
Taxa bruta d'emigració 2014\footnote{Taxes per cada 1000 habitants.} &
32.1 & 33.3
\\\addlinespace
\\\addlinespace
Índex de dependència demogràfica\footnote{Calculat pel servei
  d'Estadística de l'Ajuntament en base a la seguent fórmula: (Població
  65 i més / població de 16 a 64) x100} & 52.1 & 53
\\\addlinespace
Menors de 14 anys (\%) & 13.3 & 12.4
\\\addlinespace
Majors de 65 anys (\%) & 0 & 0
\\\addlinespace
\\\addlinespace
\bottomrule
\end{longtable}

\subsection{Nivell educatiu}\label{nivell-educatiu}

\begin{longtable}[c]{@{}lll@{}}
\toprule\addlinespace
\textbf{\ldots{}} & \textbf{al districte} & \textbf{a la ciutat}
\\\addlinespace
\midrule\endhead
\\\addlinespace
Estudis primaris (\%) & 25.2 & 22.6
\\\addlinespace
Estudis secundaris (\%) & 41.7 & 40.3
\\\addlinespace
Estudis universitaris (\%) & 19.3 & 23.9
\\\addlinespace
\\\addlinespace
\bottomrule
\end{longtable}

Nota: Moltes estadístiques oficials es calculen en base a enquestes que
no són representatives, sovint ni tant sols a nivell de districte. Per
exemple, la taxa d atur, que és el percentatge d'aturats sobre la
població activa, no es pot estimar a nivell de barri perquè la població
activa es calcula en base a la EPA, que només és representativa a nivell
de ciutat. En aquestes circunstàncies només es pot fer una aproximació
en base al nombre d'aturats i el nombre de persones en edat de
treballar. Un altre cas similar és la taxa de risc de pobresa, que es
calcula en base a l'enquesta de condicions de vida de tota la ciutat.

\newpage

\section{\textbf{Dades subjectives}}\label{dades-subjectives}

La informació que apareix a continuació ha sigut recollida mitjançant
enquestes fetes a través d'internet
(\href{http://goo.gl/OfqwdC}{\url{http://goo.gl/OfqwdC}}). Tot i que
l'enquesta segueix en funcionamenta, les dades que aquí es presentan van
ser descarregades el 28 de gener del 2015, després de més d'un mes de
funcionament. Aquesta descàrrega quan encara està en funcionament té per
objectiu el poder aportar informacions que puguin ser d'utilitat pel
diagnóstic dels barris uq està fent Guanyem. És molt important ser
conscients de que \textbf{la informació que presentem no té cap
significació estadística} (i menys a nivell de barris) \textbf{que
permeti fer inferències amb un mínim de certesa}.

En pràcticament tots els casos, la mostra aconseguida no és encara gaire
gran. A més, aquesta mostra no és mai, com veurem a continuació,
aleatòria. És per això que demanem NO pendre les informacions recollides
en aquesta secció com a dades representatives del conjunt de la població
de barris o districtes, sinó més aviat com a opinions puntuals a títol
personal. Per descomptat, aquests apunts sobre el conjunt de la mostra
no lis treuen importància a cadascuna de les valuoses opinions
recollides, les quals, per si mateixes, també expliquen bona part de la
realitat del barri o districte (i per això les oferim).

\subsection{Enquestes al districte de `SANT
MARTÍ'}\label{enquestes-al-districte-de-sant-marti}

La mostra consta de \textbf{108 enquestes} realitzades \textbf{des de 72
IPs diferents}. Com es pot comprovar a continuació, el perfil de les
persones enquestades està esbiaixat i no és representatiu del conjunt de
la població del barri o districte:

.

\includegraphics{fitxes_dels_barris_files/figure-latex/INTERVIEW_PLOTS-1.pdf}

\newpage

\subsection{Opinions sobre la situació general al barri o
districte}\label{opinions-sobre-la-situacio-general-al-barri-o-districte}

.

\includegraphics{fitxes_dels_barris_files/figure-latex/HOUSING_OPINION_PLOTS-1.pdf}

Altres problemes del barri/districte relacionats amb l'habitatge i
enumerats durant les enquestes (sense filtratge previ): \emph{atur i
precarietat / manca d'ajudes / maleït 22@ / cap / pisos turístics a
preus desorbitats, naus industrials sobrepoblades i en condicions
infra-humanes\ldots{}s / el problema al meu barri no el sé, però el que
veig és que cada cop hi ha més gent sense sostre!! / condiciones
fiscales desfavorables. / gentrificació, elevat preu, venta de pisos a
estrangers adinerats / no lo se / que el preu està molt per sobre del
preu mde cost}.

.

\includegraphics{fitxes_dels_barris_files/figure-latex/EDUCATION_OPINION_PLOTS-1.pdf}

.

Altres problemes del barri relacionats amb l'educació i enumerats durant
les enquestes (sense filtratge previ): \emph{dessanim dels professorat
que es tradueix en poca implicació / no ho sé / ensinistrament,
adoctrinament, informació nociva per als individus i en contra de les
necesitats creatives / cap, esta be / cap / pocs recursos humans i
materials per portar la classe correctament / no estem ensenyant les
necessitats pel futur: tecnologia i filosofia / no tenen cura dels
diferents tipus de cultura de cada nen. hi ha nens que queden
endarrerits / aplicació del metòde empresarial a les escoles de
primària? depriment\ldots{} / equidad relacion inmigrantes - autóctonos
/ recortes / mala qualitat del menjar dels menjadors escolars. /
desconeixo / muy mala previsión de plazas para centros de secundaria.
pocos maestros con dificultades para cubrir bajas, etc / manca de
transparencia sobre el model educatiu a les escoles / els mestres estan
molt cremats / no fills amb edad escolar / soc jubilat / poca
profesionalitat dels docents. molts han acabat fent magisteri sense
sentir-ho. / no sé, no tinc fills}.

.

\includegraphics{fitxes_dels_barris_files/figure-latex/HEALTH_SERVICE_OPINION_PLOTS-1.pdf}

.

Altres problemes del barri relacionats amb l'atenció sanitària pública i
que també han sortit (sense filtratge previ): \emph{el cap maragall es
el q te mes pressio assistencial de la ciutat / manca de professionals /
cap / professionals volcats amb manca de recursos / la revisió mèdica
anual o cada 2 anys dels nens no és completa / problemes d'organització
/ cap, estic contenta / cap / cap / cap / manca de seguiment i descarten
malaltia molt ràpid / no parece haber un problema de capacidad
profesional sino de actitud profesional. demasiada displicencia médica.
/ no l'espera d'urgències (ja s'encarreguen de dir-te que no hi vagis
encara que sigui una mica greu) sino d'alltres coses que no m'afecten,
com dares d'intervencions, etc. / escasa atención pediatrica en un
barrio con gran población infantil: revisiones periódicas cada 2 años /
quan algun especialista del cap es posa malalt/a, no hi assignen cap
substitut / hi ha una maca d'atencións terapeutiques gratuites a la
sanitat pública (logopedia, psicologia, fisioterapia, terapia
ocupacional\ldots{})) / no ho se. no fsig servir seguretat social /
temps d´espera fins que et pot veure un especialista}.

.

\includegraphics{fitxes_dels_barris_files/figure-latex/TRANSPORT_AND_POLLUTION_OPINION_PLOTS-1.pdf}

.

\includegraphics{fitxes_dels_barris_files/figure-latex/NOISE_AND_TURISM_OPINION_PLOTS-1.pdf}

.

\includegraphics{fitxes_dels_barris_files/figure-latex/TURISME_2_AND_PLACE_ABSENCE_OPINION_PLOTS-1.pdf}

.

Altres incovenients, relacionats amb el turisme, i que afecten al barri
(sense filtratge previ): \emph{estem a temps de fer prevenció respecte
al turisme / que no se gestiona bien, con una buena gestión el turismo
seria positivo. / ``proliferació del espais turística que acaben amb el
patrimoni històric del barri. / l'ús del turisme per''``transformar''"
el barri de forma que els preus s'eleven i es deplaça sutilment als
habitants de rendes mes baixes\ldots{}" / ``poca vida''``real''" en
algunes zones." / ``tinc la sort de que a nivell de carrer el meu barri
de poblenou està en millor situació que altres. però, el nivell de
turisme a altres punts de la ciutat és excessiu i perjudicial. aquí els
hotels están a la zona amplia de diagonal mar i no es nota. he marcat
brutícia perquè a la platja si ho notem. / disminución de la calidad de
los servicios: aumento restauración de''``paella y sangria''" / tots els
citats en les opcions" / bruticia a les platges / no hi ha turisme / no
hi ha turisme al meu barri. / saturació del transport públic}

Altres incovenients, relacionats amb el turisme, i que afecten al barri
(sense filtratge previ): \emph{residències i centres de dia / centres
civics publics / restaurants, cafés (oferta variada) / carrils bici de
mar a muntanya i muntanya a mar / aparcaments públics / no pensem amb la
gent gran i aquestbarri esta ple: falten residencies i centres de dia /
espais autogestionats, espai gratuir per la ciutadania, que no hagis de
pagar ni haver de ser una entitat, simplement un grup persones que ens
reunim per un objectiu comú / espais per a gossos / pacificació del
transit en general / mes horts uebans / més comerç de proximitat / falta
de plazas de aparcamiento. cada año se recortan más y más. / el meu
barri encara s'hi salva. altres hi ha més feina a fer en això.
m'agradaria que les biblioteques que com les del meu barri que tanquen a
les 20.30 obrin l'aula d'estudi a les 20.30 com abans\ldots{} és
incomodíssim la mitja hora tirada al carrer fred perquè ara obren a les
21.00. / mes places als centres esportius, i preus accesibles / zones
per gossos grans i amb tanca elevada / parcs especials per a amos i
gossos que passen una mitja de 2,5h diàries als parcs / equipaments per
gent gran / no falta ningun / equipaments culturals o millora dels
existents / vida entreveins es a dir falta vida de barri / casals
d'infants i de joves, escola de música per joves. / casal per a joves}

.

\includegraphics{fitxes_dels_barris_files/figure-latex/PARTICIPATION_OPINION_PLOTS-1.pdf}

.

\includegraphics{fitxes_dels_barris_files/figure-latex/COUNCILOR_ELECTION_OPINION_PLOTS-1.pdf}

.

\textbf{Quins altres problemes no esmentats en el qüestionari són
importants al barri on vius? I per la ciutat? (respostes en brut, sense
filtrar):} \emph{més protagonisme dels infants més petits (0-6 anys) /
el casal feixista tramuntana especialment massa proper al meu barri /
manca d'una forta xarxa de serveis socials accesible. / excès de trànsit
de vehicles / manca de participació ciutadana / manca de transparencia,
traiciona la confiança / punt de trobada per fer de tot / turismo masivo
en otros barrios (gracia, barceloneta, gotic) / el control del petit
comerç. hi ha molts forns i fruiteries. falta oferta de restauració. /
potenciar els comerços en els baixos dels edificis que actualment estan
buïts / coloms / la atenció sanitaria a inmigrants i l' atencio i
seguiment a persones grans que viuen soles / costa entrar a un teixit
associatiu, sembla tancat. / la brutícia, , la convivència amb tants
gossos, massa trànsit / dificultats a l'hora d'brir un local/negoci /
manca d'espais i refugis per animals / manca d'aparcaments i zones
verdes / la suciedad, la falta de civismo de los propietarios de perros
/ transport public / inseguretat ciutadana / seguretat i contaminacio /
que hi hagi mes civisme / limpieza / neteja, manteniments dels parcs i
jardins / arreglar les rajoles del terra del carrer rogent / dificultat
aparcament / especulació immobiliaria, moving, actitut violenta dels
cossos de policia, poques ajudes a la producció i consum sostenible,
pocs espais de participació ciutadana, el disseny del barri no es fa
tenint en compte les necessitats del veinatge, el veinatge no te
capacitat de decisió sobre el seu barri\ldots{} / atur / seguretat
ciutadana / molts dels petits comerços estan passant a mans de xinesos,
qui acomiada als treballadors o be canvia les seves condicions laborals
a unes molt pitjors. / l'empobriment en general, cada cop més gent amb
necessitats bàsiques no cobertes (aliments, energia..) / treballs de
qualitat que permetin conciliar / la neteja de les voreres i zones
ajardinades / falta potenciar la botiga petita. estan tancant totes!!! /
i·luminació a les nits, seguretat, problemes de trànsit, més transport
públic / transport / la suciedad. la basura. la limpieza del barrio y
los parques es casi nula / falta de seguridad: gran aumento de robos en
viviendas y robo callejero. / a altres barris sí que falten més entitats
culturals, cooperatives, etc. i al meu i a tots, trobo a faltar impuls i
ajude a comerços ecològics de barri i a cooperatives, per a que poc a
poc entre tots aquesta opció tingui el preu normal que ha de tenir, no
encarit per les trabes que es troben. al meu barri seria fàcil, estan
obrint petits comerços així amb ganes, però no sé si sobreviuran i no
tindrem calers per a comparhi a diari com volem per a la nostra salut i
per a la sostenibilitat. / per la ciutat: la corrupció, la privatització
de l'espai públic, l'excés d'hotels, la desatenció al ciutadà. /
comvivencia, sorolls a les places de nit, bicicletes per les aceres,
peatons pe arril bici etc / desaparició del comerç tradicional / locals
ilegals de compra/venta ferralla. / parcs disenyat per anar amb gos,
circulació amb cotxe / la pobresa / la brutícia que originen molts
incívics/es / millors mitjans de transports. la xarxa de transports
públics barcelonina és excessivament radial. / la brutícia que generen
els diferents negocis de recollida de ferralla i els inmigrants que s'hi
dediquen sense cap control a la zona de pere iv amb agricultura / les
cacas de gos als carrers i parcs / l'excesiva pavimentació del sòl,
volem espais amb sorra, jardineres a les voreres, arbres sense quitrà.
el soroll del transit. tancament del cie. deixar de criminalitzar la
població estrangera. / neteja dels carrers, estat d'alguns carrers /
visió del barri / poder fer més activitats en el centre civic /
seguretat ciutatana / neteja carrers / la falta de neteja i l'estat dels
carrers. / higiene / preu de la formació i les activitats de lleure és
altíssim, no podem pagar activitats artístiques, esportives i culturals:
esports, música, dansa,\ldots{} / que es poguesin controlar els serveis
publics com l'aigua, gas, electricitat i telèfon / el casal tramuntana
del barri de sant martí. / seguretat a nivell de robatoris menors amb
normatives més dures per a lladres sense papers i control sobre les
plaçes d´aparcament assignades a gent amb minusvalies que en moltes
ocasions son inaceptables ja que les persones a les que li han sigut
adjudicades fan un us incorrecte.. / manca de manteniment al barri:
paviment trencat, zones brutes,\ldots{} manca de serveis comunitaris
pels veïns i espais verds. no hi ha gens d´inversió perquè la gent pugui
gaudir de l´espai públic. s´està degradant molt l´aspecte dels barris,
no es cuiden els espais, ni el petit comerç i els barris no cèntrics
s´estàn convertint en dormitoris degradats. d´altra banda, per a la
ciutat, l´excés de turisme i la inversió que fa la ciutat en aquest
sentit, deixant de banda les veritables necessitats de la gent. el
centre de barcelona ja sembla lloret a l´agost i els bàrris del voltant,
sembla que no pertanyin a aquesta ciutat tan cool dels catàlegs.}.

.

\textbf{Tens alguna proposta per al barri/districte que es pogués
implementar en el futur? (respostes en brut, sense filtrar):}
\emph{espais familiars, ludoteques / tancar el casal feixista tramunana
/ peatonitzar tots els trams de carrers possibles / implantar una
conexió a nivell de carrers per a una millor conexió de les zones del
barri / liberalització de les vivendes buides per a l'us de la gent amb
necessitat. control de les residencies d'avis per a destapar el
maltracte i la falta de cura. / mes locals associatius lliures gratuits
publics / que els representants visquin al barri / carrils bici: mes i
ben fets per la seguretat de tots / fer un aparcament públic subterrani
de lloguer barat / més carril bici / ``la obligatorietat de rendir
comptes a nivell municipal i si no es cumpleix fora. així ara tindriem
la residencia alchemika en funcionament / si, can miralletes sigui un
espai per la ciutadania, per crear alternatives oci i culturals,
potenciar les capacitats de la gent, fer activitats intergeneracionals,
/ apoderar a la gent / que estiguin una miqueta més pel be dels
comerciants, a banda de la gent del barri. l'ajuntament és una mica
pasota. / fer mes mercats d'intercanvi / mas limpieza. mas control del
gasto publico / transparencia en gestion y destino de fondos / transit i
matricules alternatives / que no haya violencia callejera / mes
ludoteques publiques per a infants / algun espai public on deixar els
nens quan surten de l'escola (ex. esplai) / millor politica d'habitatge
/ implementar un proces de diagnosi comunitaria per tal de saber quines
necessitats hi ha al barri, enint en compte a tot el veinatge i també a
les entitats, associacions i grups autogestionats que treballen en pro
del barri i miren de cobrir necessitats de les persones que hi viuen.
posterior modificació de les legislacions i lleis actuals per tal
d'oferir al veiantge poderreal de decisió establint-ne les capacitats i
les limitacions. finalment aprovar pressuposto i documents que
reparteixin les responsabilitats dels canvis que s'han de dur a terme.
implementar aquest proces com a qelcom periodic, per educar al veinatge
sobre aquesta capacitat de decisió i finalment incorporar-lo al''``dia a
dia''" de les persones que hi viuen com a un procés continu, unitari i
transversal. / wifi a zones publiques com parcs o places / equilibri
oferta comercial / carrer rogen com a carrer comercial variat; massa
llicencia de forns; mesures perquè la gent no hagi d'obrir les
escombraries (veus com va una persona darrere l'altre); regular les
``''perruqueries xines``'' que són prostíbuls amagatas. / que els veïns
puguessin formar part dels plenaris de districte amb capacitat de gestió
i de decisions" / zones de gossos petits, per no haver de barrejar-los
amb els grans que els poden fer mal, com a nova york / desenvolupar
alternatives al turisme / poques arbres, pocs comercios / ``más huertos
autogestionados y más opciones de guardería pública / habitualmente, en
los bloques de vivienda toman las decisiones sólo los propietarios. los
inquilinos deberíamos tener posibilidad de participar en cuestiones
relativas a la seguridad y la vida cotidiana. actualmente, los
inquilinos en muchos sentidos estamos marginados. / no, només fa un any
que hi visc i encara no m'ho conec prou. / dada la gran población
infantil en el barrio, sería adecuada la creación de una escuela
municipal de música / restrictions a les places hoteleres i pisos
turístics / fer fora tots els ocupant''``il.legals''" de locals / si, un
parc per a anar amb gossos / espai soci cultural i esportiu al solar de
espronceda/camí antic de valència / descentralització administrativa, de
manera que es puguin prendre decisions vinculants a nivell de barri.
pressupostos participatius. reconversió de les antigues naus industrials
en espais per l'ús público-comú (finalitats culturals, associatives,
lúdiques, etc.)." / podria ser / no / treure la pavimentació i posar més
sorra / allargar la l4 (metro) fins a sant andreu i fer el pont entre
rambla prim i 11 stembre / millorar la visió del barri a la ciutat / ara
mateix no / algún servei de prevenció / detecció de sexualitat (tipus
checkpoint). el servei de l'ambulatòri en qüestions de sexualitat cal
reforçar-lo amb teixit associatiu. / no / no / més seguretat / tallers
de formaciö professional i cooperatives ecologiques. solucions km 0 pels
arurats. banc del temps? / allargar la linea 4 del metro fins a sant
andreu i fer el pont a continuació de prim per enllaçar a st. andreu /
que les escoles concertades no rebin ajudes. o privades o pùbliques. no
és de rebut pagar amb els nostres impostos escòles privades. / per mí,
el tema prioritari és que s´ha de tancar el casal tramuntana. és un
perill per la bona convivència als barri.}.

\newpage

\section{\textbf{Perfil del districte de `SANT
MARTÍ'}}\label{perfil-del-districte-de-sant-marti}

\subsection{Data: 31 de Gener del
2015.}\label{data-31-de-gener-del-2015.}

\subsection{Versió: 01 de Febrerer del
2015.}\label{versio-01-de-febrerer-del-2015.}

.

.

.

Aquest informe ha estat desenvolupat pel \textbf{Grup d'Anàlisi} de
Guanyem Barcelona.

La llicència que dóna cobertura a l'informe és una llicència
\emph{Creative Commons} (CC) del tipus BY-SA. Des del \emph{Grup
d'Anàlisi de Dades} de Guanyem creiem que tota persona ha de ser lliure
de reproduir, millorar i supervisar tot tipus de conclusió basada en
l'anàlisi de dades quantitatives i per aquesta raó optem per distribuir
el nostre material sota una llicència de copyleft. La democràcia comença
en la transparència, també a nivell metodològic, eś per això que
t'encoratgem a copiar, difondre i enriquir la informació present en
aquest treball. Sense importar si tu obtindràs un benefici econòmic o
no, l'únic que et demanem es que també facis ús d'una llicència
CC-BY-SA. Aquesta llicència implica la correcta citació i enllaç a la
font i autor del contingut utilitzat. Si no vols acollir-te a aquesta
llicència i vols fer-ne un us directa o indirectament comercial, si us
plau posa't en contacte amb nosaltres per que te n'autoritzem. Visita
\url{http://cat.creativecommons.org} per a més informació.

.

.

.

.

\includegraphics{/home/jose/Documents/GitHub/AnalisiGuanyem/Barcelona_Database_Apps/fitxes_dels_barris/guanyem.png}

\subsection{\textbf{Grup d'Anàlisi de
Dades}}\label{grup-danalisi-de-dades}

\subsection{\href{mailto:analisi.guanyem@gmail.com}{analisi.guanyem@gmail.com}}\label{analisi.guanyemgmail.com}

.

.

.

.

.

.

\includegraphics{/home/jose/Documents/GitHub/AnalisiGuanyem/Barcelona_Database_Apps/fitxes_dels_barris/creativecommons.png}\emph{Grup
d'Anàlisi de Dades Guanyem, 2015.}

\emph{© Aquesta obra està subjecta a la llicència de
Reconeixement-CompartirIgual 4.0 Internacional Creative Commons. Per
veure una còpia de la llicència, visiteu
\url{http://creativecommons.org/licenses/by-sa/4.0/}.}

\end{document}
